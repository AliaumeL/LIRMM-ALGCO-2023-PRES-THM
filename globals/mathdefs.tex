%
% Here are the definitions
% of mathematical macros.
%
%

% DEFINITIONS
\NewDocumentCommand{\defined}{}{\mathrel{:=}}
\NewDocumentCommand{\otherwise}{}{\text{otherwise}}

% BOOLEANS
\NewDocumentCommand{\Bool}{}{\mathbb{B}}
\NewDocumentCommand{\ind}{ m }{\mathbf{1}_{#1}}

% INTEGERS
\NewDocumentCommand{\divides}{}{\mathrel{|}}

% FUNCTIONS
\NewDocumentCommand{\support}{m}{\operatorname{support}(#1)}

% MSO, FO, LOGIC
\NewDocumentCommand{\MSO}{}{\mathsf{MSO}}
\NewDocumentCommand{\FO}{}{\mathsf{FO}}

% NUMBERS
\def\Nat{\mathbb{N}}
\def\Rel{\mathbb{Z}}
\def\Rat{\mathbb{Q}}
\def\Reals{\mathbb{R}}
\def\Cpx{\mathbb{C}}

% SETS
\NewDocumentCommand{\set}{ m }{\{ {#1} \}}
\NewDocumentCommand{\setof}{ m O{\mid} m }{\{ {#1} #2 {#3} \}}
\NewDocumentCommand{\seqof}{ m m }{\left({#1}\right)_{#2}}
\NewDocumentCommand{\cardinal}{ m }{\mathop{\#}\left[#1\right]}

% PERMUTATIONS
\NewDocumentCommand{\perm}{}{\sigma}
\NewDocumentCommand{\Perms}{ m }{\mathsf{Permutations}(#1)}

% AUTOMATA
\NewDocumentCommand{\words}{ O{\Sigma} }{{#1}^\star}
\NewDocumentCommand{\mono}{O{M}}{#1}
\NewDocumentCommand{\mappend}{O{M}}{\mathbin{\cdot_{#1}}}
\NewDocumentCommand{\mconcat}{O{M}}{\mathop{\pi_{#1}}}

% WORDS
\undef\Alph
\NewDocumentCommand{\Alph}{O{\Sigma}}{#1}
\NewDocumentCommand{\AGlue}{O{P}}{\mathcal{#1}}
\NewDocumentCommand{\wleq}{O{\leq}}{%
    #1^\star%
}
\NewDocumentCommand{\gwleq}{O{\leq} O{\AGlue}}{%
    #1_{#2}^\star%
}
\NewDocumentCommand{\size}{ O{} m}{ \left| #2 \right|_{#1}}

% POLYREGULAR
\NewDocumentCommand{\Npoly}{ O{} }{\Nat\mathsf{P}_{#1}}
\NewDocumentCommand{\Zpoly}{ O{} }{\Rel\mathsf{P}_{#1}}

\NewDocumentCommand{\Nsf}{ O{} }{\Nat\mathsf{SF}_{#1}}
\NewDocumentCommand{\Zsf}{ O{} }{\Rel\mathsf{SF}_{#1}}

% FACTORISATION FORESTS
\NewDocumentCommand{\Forests}{ O{} m }{\mathcal{F}_{#1}(#2)}
\NewDocumentCommand{\forestBinaryProduct}{}{\mathrel{\binaryForestProductSymbol}}
\NewDocumentCommand{\forestIdemProduct}{}{\mathop{\wordForestProductSymbol}}
\NewDocumentCommand{\forestLeaf}{ m }{\underline{#1}}
\NewDocumentCommand{\forestSem}{ m }{ \llbracket #1 \rrbracket }
\NewDocumentCommand{\forestFlatten}{ }{ \operatorname{flatten}}
\NewDocumentCommand{\forestLeq}{ }{\preceq}

\def\withForests{%
    \def\fep{\forestIdemProduct}%
    \def\fbp{\forestBinaryProduct}%
    \def\flea{\forestLeaf}%
    \def\fsem{\forestSem}%
    \def\fflat{\forestFlatten}%
}

% FOREST PATTERN
\NewDocumentCommand{\tuplePattern}{ O{F} m }{\operatorname{pat}_{#1}(#2)}
\NewDocumentCommand{\Patterns}{ O{} m }{\mathsf{Pat}_{#1}(#2)}
\NewDocumentCommand{\patternBinaryProduct}{}{\mathrel{\binaryForestProductSymbol}}
\NewDocumentCommand{\patternIdemProduct}{ m m }{\mathop{\wordForestProductSymbol}\left(#1 \mid #2\right)}
\NewDocumentCommand{\patternLeaf}{ m }{\underline{#1}}
\NewDocumentCommand{\patternSem}{ m }{ \llbracket #1 \rrbracket }

\def\withPatterns{%
    \let\pep\patternIdemProduct%
    \let\pbp\patternBinaryProduct%
    \let\plea\patternLeaf%
    \let\psem\patternSem%
    \let\pderiv\patternDeriv%
}

% LINEAR COMBINATIONS
\NewDocumentCommand{\Span}{ O{\Rel} m }{\mathsf{Lin}_{#1}\left(#2\right)}
\NewDocumentCommand{\LinPatterns}{ O{} m }{\mathsf{LinPat}_{#1}(#2)}

% LINEAR COMBINATION OF FORESTS PATTERNS
\NewDocumentCommand{\patternDeriv}{ m }{\mathop{\partial}#1}
\NewDocumentCommand{\patternCount}{ m }{\mathop{\#}\left[#1\right]}
